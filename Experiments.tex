\section{Experiments}
\label{sec:Experiments}

The three behaviors were implemented on a set of twelve 3D M-Block modules. The control system which is running the experiments is described in Figure~\ref{fig:electroncsChart}.

\begin{figure}[ht]

	%% Figure of electronics Diagram
\tikzset{box/.style={draw, rectangle, rounded corners, thick, node distance=7em, text width=6em, text centered, minimum height=3.5em}}
\tikzset{container/.style={draw, rectangle, rounded corners, dashed, inner sep=1em}}
\tikzset{line/.style={draw, thick, -latex'}}

\begin{tikzpicture}[auto]
	\node[anchor=south west,inner sep=0, minimum width = \linewidth, minimum height = 8.5cm] (emptyBox) at (0,0) {};
	\begin{scope}[x={(emptyBox.south east)},y={(emptyBox.north west)}]
	\tikzset{>=latex}
		\coordinate (leftCenter) at (0.12, 0.75);
		\coordinate (rightCenter) at (0.6, 0.85);
		\coordinate (routePt) at (0.50, 0.75);
		
		% Main Board
		\node [box] (mainBoard) at (leftCenter) {High Level Control Board (ESP8266)};
		
		% Kyles Boards...
	%	\node [box] at ([shift = ({225:3.5 cm})] mainBoard) (kb) {Motor and Power Board (nRF51422)};
	%	\node [box, right = 1.5cm of kb] (db)  {Actuator Control Board};
		
		% Face Boards
		\node [box] (f0) at (rightCenter) {Face 0 PCB  (IO~Expander)};
		\node [box, below = 0.25cm of f0] (f1) {Face 1 PCB   (IO~Expander)};
		\node [below = 0.75cm of f1] (f2) {...};
		\node [box, below = 0.75cm of f2] (f5) {Face 5 PCB   (IO~Expander)};
		
		%% I2C Connections
		\path [line, <->, orange, ultra thick] (mainBoard) -- (routePt)-- (f0);
		\path [line, <->, orange, ultra thick] (mainBoard) -- (routePt)-- (f1);
		\path [line, <->, orange, ultra thick] (mainBoard) -- (routePt)-- (f5);
		
	%	\path [line, <->, orange, ultra thick] (kb) -- (db);
		
		%% Serial Connections
		
	%	\path [line, <->, blue, ultra thick] (kb) -- (mainBoard);

		%% Inputs
		\node [align = center] (input1) at (0.91, 0.45) {Environmental \\ and \\ Neighbor \\ sensing};
	%	\node [align = left] (input2) at (0.16, 0.96) {wifi commands};

	%	\node [align = left] (input3) at (0.25, 0.24) {Motor \\ Batteries \\ SMA wire};
	%	\node [align = left] (input4) at (0.4, 0.24) {Brake \\ Em coil};
		
%		\draw [dashed,  <->] (kb.south) to [out = 270, in = 180] (input3.west);
%		\draw [dashed,  <->] (input2.south) to [out = 270, in = 235] (mainBoard.west);
%		\draw [dashed,  <->] (input4.east) to [out = 0, in = 325] (db.east);
		
		\draw [dashed,  ->] (input1.north) to [out = 90, in = 0] (f0.east);
		\draw [dashed,  ->] (input1.north) to [out = 90, in = 0] (f1.east);
		\draw [dashed,  ->] (input1.south) to [out = 270, in = 0] (f5.east);
		
		% Dashed Boxes
	%	\node[container, fit = (mainBoard) (kb) (db) (input4) (input3)] (core) {};
	%	\node at (core.south) [below, node distance = 0 and 0] {\textbf{Core}};

		\node[container, fit = (f0) (f5)] (frame) {};
		\node at (frame.south) [below, node distance = 0 and 0] {\textbf{Frame}};

%    	\node [box] (planning) {Planning};
%   	 	\node [box, below of=planning, maroon, inner sep = 0pt] (resources) {Resources};
%   	 	\node [box, below of=resources] (sensors) {Sensors};
%    	\node [box, below of=sensors] (processing) {Processing};
%    	\node [box, below of=sensors] (processing) {Processing};
%    	\node [box, below of=sensors] (processing) {Processing};
%
%    	\coordinate (middle) at ($(resources.west)!0.5!(sensors.west)$);
%    	\node [box, above of=middle, node distance=4 cm] (archive) {Archive};
%    	\node [box, below of=archive, node distance=4 cm] (reporting) {Reporting};
%
%    	\node[container, fit=(resources) (sensors)] (or) {};
%    	\node at (or.north west) [above right,node distance=0 and 0] {OR};
%
%	    \node[container, fit=(archive) (reporting)] (his) {};
%    	\node at (his.north west) [above right,node distance=0 and 0] {HIS};
%
%    	\path [line, line width = 0.5mm, red] (planning) -- (resources);
%    	\path [line] (resources) -- (sensors);
%    	\path [line] (sensors) -- (processing);
%
%    	\path [line] (archive) |- (planning);
%    	\path [line] (archive) |- (processing);
%    	\path [line] (processing)--($(processing.south)-(0,0.5)$) -| (reporting);
%
%    	\draw [line] ($(processing.south)-(0,0.5)$) -- ++(4,0) node(lowerright){} |- (planning.east);
%    	\draw [line] (lowerright |- or.east) -- (or.east -| resources.south east);
%
%    	\draw[line] (archive.170)--(reporting.10);
%    	\draw[line] (reporting.350)--(archive.190);
    \end{scope}
\end{tikzpicture}


%
%
%	\begin{tikzpicture}
%	\node[anchor=south west,inner sep=0] (image) at (0,0) {\includegraphics[width=0.9\textwidth]{some_image.jpg}};
%	\begin{scope}[x={(image.south east)},y={(image.north west)}]
%
%	\end{scope}
%	\end{tikzpicture}
%\end{document}
	
\caption{This figure illustrates the various avenues of information exchange between the different elements of the experimental setup. The \emph{orange} arrows represent bi-directional WiFi messages sent between the immobile \emph{Server Module} and the active modules, with the dashed line showing a potentially faulty connection. The \emph{blue} arrows show the simple light-based messages which active modules are able to send to neighbors which they are directly connected to. The \emph{purple} arrows represent the reading of \tagNamePlural~ between an active module and a valid and properly connected \tagName. Note that \tagNamePlural~can be read from both passive or unresponsive modules.}
	
	\label{fig:electroncsChart}
\end{figure}

\begin{table}[h]
	\caption{Experimental results for informal experiments testing the three behaviors. See the supplementary materials for a sample of the videos of the experiments.}
	
	\begin{tabular}{ p{2.4cm}  p{1.8cm}  p{1.4cm} p{1.3cm} }
		\hline
								& Path Following& Aggregation 	& Line		\\
		\hline
		Experiments				&  15 			& 3				& 11 		\\
		Successful moves		&  104			& -				& 81			\\
		Failed moves			&  15			& -				& 10			\\
		Failed tag readings		&  3			& -				& 4			\\
		Reached goal				&  16/23		& 15 / 30		& 64/84		\\

		
		
		
		
		
		
	\end{tabular}
	
	\label{tab:info}
\end{table}

%%%%%%%%%%%%%%%%%%%%%%%%%%%%%%%%%%%%%%%%%%%%%%%%%%%%%%%%%%%%%%%%%%%%%%%%%%%%%%%%%%%%%%%%%%%%%%%%%%%%%%%%%%%%%%%%%%%%%%%%%%%%%%%%%%%%%%%%%%%%%%%
%%%%%%%%%%%%%%%%%%%%%%%%%%%%%%%%%%%%%%%%%%%%%%%%%%%%%%%%%%%%%%%%%%%%%%%%%%%%%%%%%%%%%%%%%%%%%%%%%%%%%%%%%%%%%%%%%%%%%%%%%%%%%%%%%%%%%%%%%%%%%%%
\subsection{Arrow Following experiments}
\label{sec:mblocksExperimentsArrow}
%%%%%%%%%%%%%%%%%%%%%%%%%%%%%%%%%%%%%%%%%%%%%%%%%%%%%%%%%%%%%%%%%%%%%%%%%%%%%%%%%%%%%%%%%%%%%%%%%%%%%%%%%%%%%%%%%%%%%%%%%%%%%%%%%%%%%%%%%%%%%%%
%%%%%%%%%%%%%%%%%%%%%%%%%%%%%%%%%%%%%%%%%%%%%%%%%%%%%%%%%%%%%%%%%%%%%%%%%%%%%%%%%%%%%%%%%%%%%%%%%%%%%
%%%%%%%%%%%%%%%%%%%%%%%%%%%%%%%%%%%%%%%%%%
This experiment tested the ability of the modules to identify and follow the "arrows" embedded in in a set of passive and temporarily disabled active modules. The experiment consisted of a single module following the algorithm presented in Section~\ref{sec:algArrow}.

\begin{figure}[h]  
	\centering
	\begin{subfigure}[b]{0.30\linewidth}
		\begin{tikzpicture}[]	
		\node[opacity = 0.95] at (0,0) {\includegraphics[width=0.97\linewidth]{figures/arrows_0.png}};
		\node[opacity = 0.5, fill = white, rounded corners] at (-0.25,-0.5) {t = 0 s};
		\end{tikzpicture}
	\end{subfigure}
	\begin{subfigure}[b]{0.30\linewidth}
		\begin{tikzpicture}[]	
		\node[opacity = 0.95] at (0,0) {\includegraphics[width=0.97\linewidth]{figures/arrows_227.png}};
		\node[opacity = 0.5, fill = white, rounded corners] at (-0.25,-0.5) {t = 227 s};
		\end{tikzpicture}
	\end{subfigure}
	\begin{subfigure}[b]{0.30\linewidth}
		\begin{tikzpicture}[]	
		\node[opacity = 0.95] at (0,0) {\includegraphics[width=0.97\linewidth]{figures/arrows_421.png}};
		\node[opacity = 0.5, fill = white, rounded corners] at (-0.25,-0.5) {t = 421 s};
		\end{tikzpicture}
	\end{subfigure}

	\begin{subfigure}[b]{0.30\linewidth}
		\begin{tikzpicture}[]	
		\node[opacity = 0.95] at (0,0) {\includegraphics[width=0.97\linewidth]{figures/arrows_660.png}};
		\node[opacity = 0.5, fill = white, rounded corners] at (-0.25,-0.5) {t = 660 s};
		\end{tikzpicture}
	\end{subfigure}
	\begin{subfigure}[b]{0.30\linewidth}
		\begin{tikzpicture}[]	
		\node[opacity = 0.95] at (0,0) {\includegraphics[width=0.97\linewidth]{figures/arrows_686.png}};
		\node[opacity = 0.5, fill = white, rounded corners] at (-0.25,-0.5) {t = 686 s};
		\end{tikzpicture}
	\end{subfigure}
	\begin{subfigure}[b]{0.30\linewidth}
		\begin{tikzpicture}[]	
		\node[opacity = 0.95] at (0,0) {\includegraphics[width=0.97\linewidth]{figures/arrows_720.png}};
		\node[opacity = 0.5, fill = white, rounded corners] at (-0.25,-0.5) {t = 720 s};
		\end{tikzpicture}
	\end{subfigure}
%	\caption{In this experiment the module moves according to \tagNamePlural~embedded in passive modules. The module performing the experiment is preloaded with the angles for the tags with unique ID numbers, and apply these offsets in order to follow the path to the goal location.}
	\caption{In this experiment a single module moves according to "arrows" embedded in the \tagNamePlural~ in passive modules. The module moves along this path until it reaches a goal location, which is implemented as a module with its face LED's turned on.}
	
	\label{fig:arrowExperiment}
\end{figure}

%%%%%%%%%%%%%%%%%%%%%%%%%%%%%%%%%%%%%%%%%%%%%%%%%%%%%%%%%%%%%%%%%%%%%%%%%%%%%%%%%%%%%%%%%%%%%%%%%%%%%%%%%%%%%%%%%%%%%%%%%%%%%%%%%%%%%%%%%%%%%%%
%%%%%%%%%%%%%%%%%%%%%%%%%%%%%%%%%%%%%%%%%%%%%%%%%%%%%%%%%%%%%%%%%%%%%%%%%%%%%%%%%%%%%%%%%%%%%%%%%%%%%%%%%%%%%%%%%%%%%%%%%%%%%%%%%%%%%%%%%%%%%%%
\subsection{Line formation experiments}
\label{sec:mblocksExperimentsLine}
%%%%%%%%%%%%%%%%%%%%%%%%%%%%%%%%%%%%%%%%%%%%%%%%%%%%%%%%%%%%%%%%%%%%%%%%%%%%%%%%%%%%%%%%%%%%%%%%%%%%%%%%%%%%%%%%%%%%%%%%%%%%%%%%%%%%%%%%%%%%%%%
%%%%%%%%%%%%%%%%%%%%%%%%%%%%%%%%%%%%%%%%%%%%%%%%%%%%%%%%%%%%%%%%%%%%%%%%%%%%%%%%%%%%%%%%%%%%%%%%%%%%%%%%%%%%%%%%%%%%%%%%%%%%%%%%%%%%%%%%%%%%%%%

These experiments aimed to transform 3D structures with several constraints: no holes, no modules connected by three or more connection faces, into a single horizontal line. These experiments were run with the aid a centralized WiFi server - whose function is to identify which module is the "seed" for the line. The server receives WiFi  status messages from the modules which includes information about their neighbors. The server attempts to find the longest existing line in the structure, and then picks one of the modules closest to the center of that line, and sends a message to that module indicating that it is the starting point for the line. 


\begin{figure}[h]  
	\centering
	\begin{subfigure}[b]{0.32\linewidth}
		\begin{tikzpicture}[]	
			\node[opacity = 0.95] at (0,0) {\includegraphics[width=0.95\linewidth]{figures/ActualLine_1.png}};
			\node[opacity = 0.5, fill = white, rounded corners] at (-0.5,-0.5) {t = 0 s};
		\end{tikzpicture} 
	\end{subfigure}
	\begin{subfigure}[b]{0.32\linewidth}
			\begin{tikzpicture}[]	
		\node[opacity = 0.95] at (0,0) {\includegraphics[width=0.95\linewidth]{figures/ActualLine_2.png}};
		\node[opacity = 0.5, fill = white, rounded corners] at (-0.5,-0.5) {t = 30 s};
		\end{tikzpicture}
	\end{subfigure}
	\begin{subfigure}[b]{0.32\linewidth}
			\begin{tikzpicture}[]	
		\node[opacity = 0.95] at (0,0) {\includegraphics[width=0.95\linewidth]{figures/ActualLine_3.png}};
		\node[opacity = 0.5, fill = white, rounded corners] at (-0.5,-0.5) {t = 60 s};
		\end{tikzpicture}
	\end{subfigure}

	\begin{subfigure}[b]{0.32\linewidth}
			\begin{tikzpicture}[]	
		\node[opacity = 0.95] at (0,0) {\includegraphics[width=0.95\linewidth]{figures/ActualLine_4.png}};
		\node[opacity = 0.5, fill = white, rounded corners] at (-0.5,-0.5) {t = 120 s};
		\end{tikzpicture}
	\end{subfigure}
	\begin{subfigure}[b]{0.32\linewidth}
			\begin{tikzpicture}[]	
		\node[opacity = 0.95] at (0,0) {\includegraphics[width=0.95\linewidth]{figures/ActualLine_5.png}};
		\node[opacity = 0.5, fill = white, rounded corners] at (-0.5,-0.5) {t = 230 s};
		\end{tikzpicture}
	\end{subfigure}
	\begin{subfigure}[b]{0.32\linewidth}
			\begin{tikzpicture}[]	
		\node[opacity = 0.95] at (0,0) {\includegraphics[width=0.95\linewidth]{figures/ActualLine_6.png}};
		\node[opacity = 0.5, fill = white, rounded corners] at (-0.5,-0.5) {t = 300 s};
		\end{tikzpicture}
	\end{subfigure}
	
	\caption{This experiment shows a 3D configuration of six 3D M-Blocks reconfiguring into a line.}
	
	\label{fig:lineExperiment}
\end{figure}

%%%%%%%%%%%%%%%%%%%%%%%%%%%%%%%%%%%%%%%%%%%%%%%%%%%%%%%%%%%%%%%%%%%%%%%%%%%%%%%%%%%%%%%%%%%%%%%%%%%%%%%%%%%%%%%%%%%%%%%%%%%%%%%%%%%%%%%%%%%%%%%
%%%%%%%%%%%%%%%%%%%%%%%%%%%%%%%%%%%%%%%%%%%%%%%%%%%%%%%%%%%%%%%%%%%%%%%%%%%%%%%%%%%%%%%%%%%%%%%%%%%%%%%%%%%%%%%%%%%%%%%%%%%%%%%%%%%%%
\subsection{Light guided aggregation experiments}
\label{sec:mblocksExperimentsLight}
%%%%%%%%%%%%%%%%%%%%%%%%%%%%%%%%%%%%%%%%%%%%%%%%%%%%%%%%%%%%%%%%%%%%%%%%%%%%%%%%%%%%%%%%%%%%%%%%%%%%%%%%%%%%%%%%%%%%%%%%%%%%%%%%%%%%%%%%%%%%%%%
%%%%%%%%%%%%%%%%%%%%%%%%%%%%%%%%%%%%%%%%%%%%%%%%%%%%%%%%%%%%%%%%%%%%%%%%%%%%%%%%%%%%%%%%%%%%%%%%%%%%%%%%%%%%%%%%%%%%%%%%%%%%%%%%%%%%%%%%%%%%%%%


These experiments implement the photo-taxis Braitenberg behavior for a group of 3D M-Block modules. The experiment includes a single light source and a special target module indicating a initial connection location. If a module connects to a valid connection point it joins the aggragate structure and then signals to other modules that it is now also part of the structure through wireless and light signals. This essentially forms a single "crystal" of aggregated modules which grows as more modules join. In these experiment the modules are gradually released into a confined environment (0.5~m x 0.5~m) which is bounded by foam padded walls. The modules move until they either exhaust their battery or connect to the designated aggregate structure.

%\newcommand{\figureWidth}{0.32\linewidth}
\begin{figure}[h]  
	\centering
	\begin{subfigure}[b]{0.32\linewidth}
		
		\begin{tikzpicture}[]	
			\node[opacity = 0.95] at (0,0) {\includegraphics[width = \linewidth]{figures/1-000.png}};
			\node[opacity = 0.5, fill = white, rounded corners] at (-0.5,-0.5) {t = 000 s};
		\end{tikzpicture}
		
	\end{subfigure}
	\begin{subfigure}[b]{0.32\linewidth}
		
		\begin{tikzpicture}[]	
		\node[opacity = 0.95] at (0,0) {\includegraphics[width = \linewidth]{figures/2-120.png}};
		\node[opacity = 0.5, fill = white, rounded corners] at (-0.5,-0.5) {t = 120 s};
		\end{tikzpicture}
		
	\end{subfigure}
	\begin{subfigure}[b]{0.32\linewidth}

		\begin{tikzpicture}[]	
		\node[opacity = 0.95] at (0,0) {\includegraphics[width = \linewidth]{figures/3-220.png}};
		\node[opacity = 0.5, fill = white, rounded corners] at (-0.5,-0.5) {t = 220 s};
		\end{tikzpicture}
		
	\end{subfigure}

	\begin{subfigure}[b]{0.32\linewidth}
		\begin{tikzpicture}[]	
		\node[opacity = 0.95] at (0,0) {\includegraphics[width = \linewidth]{figures/4-320.png}};
		\node[opacity = 0.5, fill = white, rounded corners] at (-0.5,-0.5) {t = 320 s};
		\end{tikzpicture}
	\end{subfigure}
	\begin{subfigure}[b]{0.32\linewidth}
		\begin{tikzpicture}[]	
		\node[opacity = 0.95] at (0,0) {\includegraphics[width = \linewidth]{figures/5-500.png}};
		\node[opacity = 0.5, fill = white, rounded corners] at (-0.5,-0.5) {t = 500 s};
		\end{tikzpicture} 
	\end{subfigure}
	\begin{subfigure}[b]{0.32\linewidth}
		\begin{tikzpicture}[]	
		\node[opacity = 0.95] at (0,0) {\includegraphics[width = \linewidth]{figures/6-600.png}};
		\node[opacity = 0.5, fill = white, rounded corners] at (-0.5,-0.5) {t = 600 s};
		\end{tikzpicture}
	\end{subfigure}
	
	\caption{This experiment illustrates the light guided aggregation behavior. The goal location and the light source are both in the upper left hand corner. If modules form aggregates that are not connected to the goal location (e.g. at t = 220s), they dissassemble and continue trying to reach the goal location.}
	
	\label{fig:LightExperiment}
\end{figure}
