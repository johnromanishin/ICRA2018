%% Figure of electronics Diagram
\tikzset{box/.style={draw, rectangle, rounded corners, thick, node distance=7em, text width=6em, text centered, minimum height=3.5em}}
\tikzset{container/.style={draw, rectangle, rounded corners, dashed, inner sep=1em}}
\tikzset{line/.style={draw, thick, -latex'}}

\begin{tikzpicture}[auto]
	\node[anchor=south west,inner sep=0, minimum width = \linewidth, minimum height = 7.5cm] (emptyBox) at (0,0) {};
	\begin{scope}[x={(emptyBox.south east)},y={(emptyBox.north west)}]
	\tikzset{>=latex}
		\coordinate (leftCenter) at (0.12, 0.65);
		\coordinate (aboveleftCenter) at (0.16, 0.9);
		\coordinate (rightCenter) at (0.55, 0.8);
		\coordinate (routePt) at (0.50, 0.75);
		
		% Main Board
		\node [box, fill = green!50] (mainBoard) at (leftCenter) {Server Module};
		
		% Kyles Boards...
	%	\node [box] at ([shift = ({225:3.5 cm})] mainBoard) (kb) {Motor and Power Board (nRF51422)};
	%	\node [box, right = 1.5cm of kb] (db)  {Actuator Control Board};
		
		% Face Boards
		\node [box, red, fill = red!10] (f0) at (rightCenter) {Unresponsive Module ID 1};
		\node [box, below = 1cm of f0] (f1) {Active Module ID 2};
	%	\node [below = 0.5cm of f1] (f2) {...};
		\node [box, below = 3cm of f0] (f5) {Active Module ID N};
		
		\node [fill = orange!50, box, left = 1.25cm of f5] (f6) {Passive Modules};
		
		%% I2C Connections
		\path [line, <->, orange, ultra thick, dashed] (mainBoard) -- (f0);
		\path [line, <->, orange, ultra thick] (mainBoard) -- (f1);
		\path [line, <->, orange, ultra thick] (mainBoard) -- (f5);
		
		%% Magnetic Tag Readings
		\path [line, <->, purple, ultra thick] (f0) to [bend left = 10] (f1);
		\path [line, <->, purple, ultra thick] (f1) to [bend left = 10] (f5);
		\path [line, ->, purple, ultra thick] (f5) -- (f6);
		
		%% Light Messages
		\path [line, <->, blue, ultra thick, dashed] (f0) to [bend right = 10] (f1);
		\path [line, <->, blue, ultra thick] (f1) to [bend right = 10] (f5);
	%	\path [line, <->, blue, ultra thick,  bend right] (f5) -- (f6);
		
		
	%	\path [line, <->, orange, ultra thick] (kb) -- (db);
		
		%% Serial Connections
		
	%	\path [line, <->, blue, ultra thick] (kb) -- (mainBoard);

		%% Inputs
		\node [align = center] (input1) at (0.85, 0.45) {Light \\ and IMU \\ sensing};
	%	\node [align = left] (input2) at (0.16, 0.96) {wifi commands};

	%	\node [align = left] (input3) at (0.25, 0.24) {Motor \\ Batteries \\ SMA wire};
	%	\node [align = left] (input4) at (0.4, 0.24) {Brake \\ Em coil};
		
%		\draw [dashed,  <->] (kb.south) to [out = 270, in = 180] (input3.west);
%		\draw [dashed,  <->] (input2.south) to [out = 270, in = 235] (mainBoard.west);
%		\draw [dashed,  <->] (input4.east) to [out = 0, in = 325] (db.east);
		
		\draw [dashed,  ->] (input1.north) to [out = 90, in = 0] (f0.east);
		\draw [dashed,  ->] (input1.north) to [out = 90, in = 0] (f1.east);
		\draw [dashed,  ->] (input1.south) to [out = 270, in = 0] (f5.east);
		
		% Dashed Boxes
	%	\node[container, fit = (mainBoard) (kb) (db) (input4) (input3)] (core) {};
	%	\node at (core.south) [below, node distance = 0 and 0] {\textbf{Core}};

		\node[container, fit = (f0) (f5)] (frame) {};
		\node at (frame.south) [below, node distance = 0 and 0] {\textbf{Active Modules}};
		
		\tikzmath{\radius = 0.5;}
		\fill[gray] (aboveleftCenter) + (0, \radius cm) arc (90:270:\radius cm);
		\fill[yellow] (aboveleftCenter) + (0, -\radius cm) arc (270:450:\radius cm);
		
		\path [green, line, ->, ultra thick] (mainBoard.north) -- (0.16, 0.84);
		% Create a red filled arc, starting 2 below the center, 
		% with a start angle of 270°, an end angle of 450° and a radius of 2
%		\fill[red] (center) + (0, -2) arc (270:450:2);
		
	%	\node[fill = yellow, above = 1 cm of mainBoard, circle] (light) {lamp};
	%	\node[fill = gray, left = 0 cm of light, minimum height = 0.5 cm] (bottom) {};
	%	\node[fill = yellow, above = 1 cm of mainBoard, circle] (light) {lamp};
		

	%	\node[fill = gray, left = 0 cm of light, minimum height = 0.5 cm] (bottom) {};

%    	\node [box] (planning) {Planning};
%   	 	\node [box, below of=planning, maroon, inner sep = 0pt] (resources) {Resources};
%   	 	\node [box, below of=resources] (sensors) {Sensors};
%    	\node [box, below of=sensors] (processing) {Processing};
%    	\node [box, below of=sensors] (processing) {Processing};
%    	\node [box, below of=sensors] (processing) {Processing};
%
%    	\coordinate (middle) at ($(resources.west)!0.5!(sensors.west)$);
%    	\node [box, above of=middle, node distance=4 cm] (archive) {Archive};
%    	\node [box, below of=archive, node distance=4 cm] (reporting) {Reporting};
%
%    	\node[container, fit=(resources) (sensors)] (or) {};
%    	\node at (or.north west) [above right,node distance=0 and 0] {OR};
%
%	    \node[container, fit=(archive) (reporting)] (his) {};
%    	\node at (his.north west) [above right,node distance=0 and 0] {HIS};
%
%    	\path [line, line width = 0.5mm, red] (planning) -- (resources);
%    	\path [line] (resources) -- (sensors);
%    	\path [line] (sensors) -- (processing);
%
%    	\path [line] (archive) |- (planning);
%    	\path [line] (archive) |- (processing);
%    	\path [line] (processing)--($(processing.south)-(0,0.5)$) -| (reporting);
%
%    	\draw [line] ($(processing.south)-(0,0.5)$) -- ++(4,0) node(lowerright){} |- (planning.east);
%    	\draw [line] (lowerright |- or.east) -- (or.east -| resources.south east);
%
%    	\draw[line] (archive.170)--(reporting.10);
%    	\draw[line] (reporting.350)--(archive.190);
    \end{scope}
\end{tikzpicture}


%
%
%	\begin{tikzpicture}
%	\node[anchor=south west,inner sep=0] (image) at (0,0) {\includegraphics[width=0.9\textwidth]{some_image.jpg}};
%	\begin{scope}[x={(image.south east)},y={(image.north west)}]
%
%	\end{scope}
%	\end{tikzpicture}
%\end{document}