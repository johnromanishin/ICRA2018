%%%%%%%%%%%%%%%%%%%%%%%%%
\section{Discussion}
\label{sec:Discussion}
%%%%%%%%%%%%%%%%%%%%%%%%%
This paper presents a new magnetic fiducial, the \tagName, for Modular Self-Reconfigurable Robots, and introduces three behaviors which utilize it to accomplish specific tasks.
% The \tagNamePlural~ offers a combination of several features that exist disparately in other MSRR identification technologies.
The \tagNamePlural's use of permanent magnets is inexpensive, functionally simple, and allows for the reading of passive or inactive modules. While other technoligies including NFC tags are also promissing for this application, they are more complex, more expensive (when considering both tags and reader) and are a proprietary standard. %Dusty, hot, and sunlight-exposed environments may be damaging to plastics and readable surfaces in optically based identification systems, and \tagName~ could offer greater robustness in these applications.
One additional advantage of \tagName~ is its scalability in modular robotics applications that involve RF communications. Many magnetic rotary position sensors are immune to stray magnetic fields, and have a very short detection range. A large number of magnet-sensor pairs can be used within a system of many modular robots without any RF interference or confusion. In contrast, RF-based technologies may interfere with one another when densely packed, or interfere with other communications devices in EM-noisy environments including outer space and industrial applications.

The experimental results and behaviors presented in this work are based on preliminary hardware, and have relatively high error rates due to manufacturing and design limitations. Additionally the 3D M-Block robots which this system is tested on has relatively unreliable movement abilities due to electronic and manufacturing problems. However we believe that this system provides justification that this technology could provide a framework that future work could follow to create an effective system to identify the configuration of and control systems with millions of modular elements.

%The behaviors presented in this work are wi applying these behaviors to future systems with millions of modules.