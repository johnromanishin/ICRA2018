%%%%%%%%%%%%%%%%%%%%%%%%%
\section{Discussion}
\label{sec:Discussion}
%%%%%%%%%%%%%%%%%%%%%%%%%
This paper presents a new magnetic fiducial, the \tagName, for Modular Self-Reconfigurable Robots, and introduces three behaviors which use this tag to accomplish various tasks. The \tagNamePlural~ offers a combination of several features that exist disparately in other MSRR identification technologies. The use of permanent magnets is inexpensive, functionally simple, and means that tags do not have to be powered. NFC passive tags offer this feature as well, but with greater complexity and higher cost when considering both tags and reader. Dusty, hot, and sunlight-exposed environments may be damaging to plastics and readable surfaces in optically based ID systems, and \tagName~ could offer greater robustness in these applications. One additional advantage of \tagName~ is its scalability in modular robotics applications that involve RF communications. Many magnetic rotary position sensors, including those used in this work, are immune to external magnetic stray fields. A large number of magnet/sensor pairs can be used in tandem within a system of many modular robots without any interferences. In contrast, RF-based technologies may interfere with one another when densely packed, or interfere with other communications devices in EM-noisy environments like outer space and certain industrial applications.

The experimental results and behaviors presented in this work are based on preliminary hardware, and have relatively high error rates due to manufacturing and design limitations. Additionally the 3D M-Block robots which this system is tested on has relatively unreliable movement abilities due to electronic and manufacturing issues. However we believe that this system provides justification that the technology could provide a framework that future work could follow to create an effective system to identify and control systems with millions of modular elements.

%The behaviors presented in this work are wi applying these behaviors to future systems with millions of modules.