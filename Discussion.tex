%%%%%%%%%%%%%%%%%%%%%%%%%
\section{Discussion}
\label{sec:Discussion}
%%%%%%%%%%%%%%%%%%%%%%%%%

\tagName offers a combination of several features that exist independently in other ID technologies. Its use of permanent magnets is inexpensive, functionally simple, and means that tags do not have to be powered. NFC passive tags offer this feature as well, but with greater complexity and higher cost when considering both tags and reader. Dusty, hot, and sunlight-exposed environments may be damaging to plastics and readable surfaces in optically based ID systems, but \tagName would offer longevity and robustness in these applications.

One of the major advantages of \tagName is its scalability in modular robotics applications that involve RF communications. Many magnetic rotary position sensors, including those used on Mblocks, are immune to external magnetic stray fields. A large number of magnet/sensor pairs can be used in tandem within a large system of modular robots. Outside of the static field of the magnet itself, these magnet/sensor pairs generate no appreciable electromagnetic pulses, and they do not aim to communicate wirelessly. In contrast, RF-based technologies may interfere with one another when densely packed, or even with other communications devices. Zigbee, older implementations of WiFi, industrial microwave ovens, commercial two-way radios, cordless telephones, and a host of other devices and technologies utilize the UHF band, which is shared by some RFID passive tags. Avoiding reliance on EM waves allows \tagName to be useful for EM-noisy environments like outer space and certain industrial applications.