%%%%%%%%%%%%%%%%%%%%%%%%%
\section{Introduction}
\label{sec:Introduction}
%%%%%%%%%%%%%%%%%%%%%%%%%

Modular self-reconfigurable robots (MSRR) have been proposed as one method to create general purpose robotic systems of arbitrary complexity in an autonomous way. MSRR systems generally can be thought of as consisting of individual \textbf{modules}, which connect to other modular elements, either active modules or passive elements, through standardized \textbf{connectors} to create a specific \textbf{configuration} in order to accomplish a designated task. Much of the work in the MSRR field has focused either on the preliminary development of novel hardware systems, or general purpose algorithms on simulated impractical systems. Few actual systems stay under active development long eneough in order to develop practical algorithms that can accomplish tasks while operating under the constraints of actual existing hardware. This paper is focused on implementing and analyzing three separate practical behaviors on 3D M-Block modules.

For large systems of module

This paper
These tags ability allows for a scalable, reliable, inexpensive, and robust way to perform identification, location, and orientation tasks. 

\begin{itemize}
	\item Path following - other text goes here
	\item Line formation - text goes here
	\item Light gradient aggregation - text
\end{itemize}

The remainder of the paper is organized as follows: Section~\ref{sec:RelatedWork} gives an overview of related
work that pertains to modular robots with a focus on how MSRR systems identify and encode physical configuration information through their connectors.
Section~\ref{sec:Hardware} starts with a brief overview of the 3D Mblock modules, and then gives a detailed description of the novel \tagName hardware and associated electronics.
%Next Section ~\ref{sec:Algorithims} presents the two algorithms which utilize the magnetic tags to 1. turn arbitrary configurations into %a line, and 2. form simple shapes from a line of modules.
Next, Section~\ref{sec:Experiments}
presents data characterizing the hardware and the results of
experiments with the system.
Finally, Section~\ref{sec:Discussion}
concludes with a short discussion and ideas for future work.



\begin{figure}[t]
	\centering
	\begin{subfigure}[b]{1.75 in}
	%	\resizebox{.48\linewidth}{0.6 in}
	%	{
			\includegraphics[width=.9\linewidth]{Figures/mTagsCover.png}
	%	}
	%	\subcaption{b.}
	\end{subfigure}
	~
	\begin{subfigure}[b]{1.5 in}


		\resizebox{1.5 in}{1.25 in}
		{
			\begin{tikzpicture}[x=(220:1cm), y=(-40:1cm), z=(90:0.707cm)]
			
			%%
%this code is from...
\setcounter{x}{0}%
\setcounter{y}{0}%
\setcounter{z}{0}%

% The angles of x,y,z-axes
\newcommand\xaxis{210}
\newcommand\yaxis{-30}
\newcommand\zaxis{90}

% The top side of a cube
\newcommand\topside[3]{
	\fill[fill=yellow, draw=black,shift={(\xaxis:#1)},shift={(\yaxis:#2)},
	shift={(\zaxis:#3)}] (0,0) -- (30:1) -- (0,1) --(150:1)--(0,0);
}

% The left side of a cube
\newcommand\leftside[3]{
	\fill[fill=red, draw=black,shift={(\xaxis:#1)},shift={(\yaxis:#2)},
	shift={(\zaxis:#3)}] (0,0) -- (0,-1) -- (210:1) --(150:1)--(0,0);
}

% The right side of a cube
\newcommand\rightside[3]{
	\fill[fill=blue, draw=black,shift={(\xaxis:#1)},shift={(\yaxis:#2)},
	shift={(\zaxis:#3)}] (0,0) -- (30:1) -- (-30:1) --(0,-1)--(0,0);
}

% The cube 
\newcommand\cube[3]{
	\topside{#1}{#2}{#3} \leftside{#1}{#2}{#3} \rightside{#1}{#2}{#3}
}

\newcommand\ArrowNE[3]
{
	\node at (#1+0.5, #2+0.5, #3) 
%	\node at (#1, #2, #3) 
	{
		\begin{tikzpicture}
		\draw[->, thick, >={Stealth[round]}, line width=0.4mm] (0.3,0) -- (-0.3,0);
		\end{tikzpicture}};
		
}

\newcommand\ArrowNL[3]
{
	\node at (#1+1, #2+0.5, #3-0.5) 
	%	\node at (#1, #2, #3) 
	{
		\begin{tikzpicture}
		\draw[->, thick, >={Stealth[round]}, line width=0.4mm] (0.16,0.16) -- (-0.16,-0.16);
		\end{tikzpicture}};
	
}

\newcommand\ArrowNR[3]
{
	\node at (#1+0.5, #2+1, #3-0.5) 
	%	\node at (#1, #2, #3) 
	{
		\begin{tikzpicture}
		\draw[->, thick, >={Stealth[round]}, line width=0.4mm] (0.16,0.16) -- (-0.16,-0.16);
		\end{tikzpicture}};
	
}

\newcommand\ArrowNW[3]
{
	\node at (#1+0.5, #2+0.5, #3) 
%	\node at (#1, #2, #3) 
	{
		\begin{tikzpicture}
		\draw[->, thick, >={Stealth[round]}, line width=0.4mm] (0,0.3) -- (0,-0.3);
		\end{tikzpicture}};
	
}

\newcommand\ArrowSW[3]
{
	\node at (#1+0.5, #2+0.5, #3) 
%	\node at (#1, #2, #3) 
	{
		\begin{tikzpicture}
		\draw[<-, thick, >={Stealth[round]}, line width=0.4mm] (0.3,0) -- (-0.3,0);
		\end{tikzpicture}};
	
}

\newcommand\ArrowSE[3]
{
	\node at (#1+0.5, #2+0.5, #3) 
%	\node at (#1, #2, #3) 
	{
		\begin{tikzpicture}
		\draw[<-, thick, >={Stealth[round]}, line width=0.4mm] (0,0.3) -- (0,-0.3);
		\end{tikzpicture}};
	
}

% Definition of \planepartition
% To draw the following plane partition, just write \planepartition{ {a, b, c}, {d,e} }.
%  a b c
%  d e
\newcommand\planepartition[1]{
	\setcounter{x}{-1}
	\foreach \a in {#1} {
		\addtocounter{x}{1}
		\setcounter{y}{-1}
		\foreach \b in \a {
			\addtocounter{y}{1}
			\setcounter{z}{-1}
			\foreach \c in {0,...,\b} {
				\addtocounter{z}{1}
				\ifthenelse{\c=0}{\setcounter{z}{-1},\addtocounter{y}{0}}{
					\cube{\value{x}}{\value{y}}{\value{z}}}
			}
		}
	}
}


%%\begin{tikzpicture}[x=(220:1cm), y=(-40:1cm), z=(90:0.707cm)]
	%\planepartition{{0,0,1},{1,1,1},{1,0,0},{1,0,1}};
\foreach \m [count=\y] in {{1,1,1,1},{1,1,1},{1,1,1,1},{1,1,1}}{
	\foreach \n [count=\x] in \m {
		\ifnum \n>0
		\foreach \z in {1,...,\n}{
			\draw [fill=blue!30] (\x+1,\y,\z) -- (\x+1,\y+1,\z) -- (\x+1, \y+1, \z-1) -- (\x+1, \y, \z-1) -- cycle;
			\draw [fill=blue!40] (\x,\y+1,\z) -- (\x+1,\y+1,\z) -- (\x+1, \y+1, \z-1) -- (\x, \y+1, \z-1) -- cycle;
			\draw [fill=blue!10] (\x,\y,\z)   -- (\x+1,\y,\z)   -- (\x+1, \y+1, \z)   -- (\x, \y+1, \z) -- cycle;  
		}
	
		\fi
	}
}   

\foreach \m [count=\y] in {{0,0,0,1}, {0,0,0}} {
	\foreach \n [count=\x] in \m {
		\ifnum \n>0
		\foreach \z in {1,...,\n}{
			\draw [fill=green!30] (\x+1,\y,\z) -- (\x+1,\y+1,\z) -- (\x+1, \y+1, \z-1) -- (\x+1, \y, \z-1) -- cycle;
			\draw [fill=green!40] (\x,\y+1,\z) -- (\x+1,\y+1,\z) -- (\x+1, \y+1, \z-1) -- (\x, \y+1, \z-1) -- cycle;
			\draw [fill=green!10] (\x,\y,\z)   -- (\x+1,\y,\z)   -- (\x+1, \y+1, \z)   -- (\x, \y+1, \z) -- cycle;  
		}
		
		\fi
	}
}   

\ArrowNW{1}	{3}	{1};

\foreach \m [count=\y] in {{0,0,0,0},{0,0,0},{0,0,0,0},{2,0,0}}{
	\foreach \n [count=\x] in \m {
		\ifnum \n>0
		\foreach \z in {1,...,\n}{
			\draw [fill=orange!30] (\x+1,\y,\z) -- (\x+1,\y+1,\z) -- (\x+1, \y+1, \z-1) -- (\x+1, \y, \z-1) -- cycle;
			\draw [fill=orange!40] (\x,\y+1,\z) -- (\x+1,\y+1,\z) -- (\x+1, \y+1, \z-1) -- (\x, \y+1, \z-1) -- cycle;
			\draw [fill=orange!10] (\x,\y,\z)   -- (\x+1,\y,\z)   -- (\x+1, \y+1, \z)   -- (\x, \y+1, \z) -- cycle;  
		}
		
		\fi
	}
}  

\foreach \m [count=\y] in {{0},{0},{0},{0,1,1}}{
	\foreach \n [count=\x] in \m {
		\ifnum \n>0
		\foreach \z in {1,...,\n}{
			\draw [fill=blue!30] (\x+1,\y,\z) -- (\x+1,\y+1,\z) -- (\x+1, \y+1, \z-1) -- (\x+1, \y, \z-1) -- cycle;
			\draw [fill=blue!40] (\x,\y+1,\z) -- (\x+1,\y+1,\z) -- (\x+1, \y+1, \z-1) -- (\x, \y+1, \z-1) -- cycle;
			\draw [fill=blue!10] (\x,\y,\z)   -- (\x+1,\y,\z)   -- (\x+1, \y+1, \z)   -- (\x, \y+1, \z) -- cycle;  
		}
		
		\fi
	}
} 

\foreach \m [count=\y] in {{0},{0},{0},{1,0,0}}{
	\foreach \n [count=\x] in \m {
		\ifnum \n>0
		\foreach \z in {1,...,\n}{
		%	\draw [fill=blue!30] (\x+1,\y,\z) -- (\x+1,\y+1,\z) -- (\x+1, \y+1, \z-1) -- (\x+1, \y, \z-1) -- cycle;
			\draw [fill=blue!40] (\x,\y+1,\z) -- (\x+1,\y+1,\z) -- (\x+1, \y+1, \z-1) -- (\x, \y+1, \z-1) -- cycle;
		%	\draw [fill=blue!10] (\x,\y,\z)   -- (\x+1,\y,\z)   -- (\x+1, \y+1, \z)   -- (\x, \y+1, \z) -- cycle;  
		}
		
		\fi
	}
} 
		"x" "y" "z"

\ArrowNR{4} {1}	{1};
\ArrowNL{4} {1}	{1};

\ArrowNR{4} {3}	{1};
\ArrowNL{4} {3}	{1};

\ArrowNR{3} {4}	{1};
\ArrowNL{3} {4}	{1};

\ArrowNR{1} {4}	{1};
\ArrowNR{2} {4}	{1};

\ArrowNL{3} {2}	{1};

\ArrowSW{1}	{1}	{1};
\ArrowSW{2}	{1}	{1};
\ArrowSW{3}	{1}	{1};
\ArrowSW{4}	{1}	{1};


\ArrowNW{1}	{2}	{1};
\ArrowSW{2}	{2}	{1};
\ArrowNW{3}	{2}	{1};

\ArrowNW{2}	{3}	{1};
\ArrowNE{3}	{3}	{1};
\ArrowNE{4}	{3}	{1};

%\ArrowNW{1}	{4}	{1};
\ArrowSW{2}	{4}	{1};
\ArrowNW{3}	{4}	{1};

			
			\end{tikzpicture}
		}

		%	\subcaption{a.}
	\end{subfigure}

		\begin{subfigure}[b]{1.75 in}
		%	\resizebox{.48\linewidth}{0.6 in}
		%	{
		\includegraphics[width=.9\linewidth]{Figures/mTagsCover.png}
		%	}
		%	\subcaption{b.}
	\end{subfigure}
	~
	\begin{subfigure}[b]{1.5 in}
		
		
		\includegraphics[width=.9\linewidth]{Figures/mTagsCover.png}
		
		%	\subcaption{a.}
	\end{subfigure}


	\caption{\tagNamePlural renderings with displaying arrows}

	\label{fig:PlaneChanging2}
\end{figure}

%\begin{figure}[htb]
%
%	%%
%this code is from...
\setcounter{x}{0}%
\setcounter{y}{0}%
\setcounter{z}{0}%

% The angles of x,y,z-axes
\newcommand\xaxis{210}
\newcommand\yaxis{-30}
\newcommand\zaxis{90}

% The top side of a cube
\newcommand\topside[3]{
	\fill[fill=yellow, draw=black,shift={(\xaxis:#1)},shift={(\yaxis:#2)},
	shift={(\zaxis:#3)}] (0,0) -- (30:1) -- (0,1) --(150:1)--(0,0);
}

% The left side of a cube
\newcommand\leftside[3]{
	\fill[fill=red, draw=black,shift={(\xaxis:#1)},shift={(\yaxis:#2)},
	shift={(\zaxis:#3)}] (0,0) -- (0,-1) -- (210:1) --(150:1)--(0,0);
}

% The right side of a cube
\newcommand\rightside[3]{
	\fill[fill=blue, draw=black,shift={(\xaxis:#1)},shift={(\yaxis:#2)},
	shift={(\zaxis:#3)}] (0,0) -- (30:1) -- (-30:1) --(0,-1)--(0,0);
}

% The cube 
\newcommand\cube[3]{
	\topside{#1}{#2}{#3} \leftside{#1}{#2}{#3} \rightside{#1}{#2}{#3}
}

\newcommand\ArrowNE[3]
{
	\node at (#1+0.5, #2+0.5, #3) 
%	\node at (#1, #2, #3) 
	{
		\begin{tikzpicture}
		\draw[->, thick, >={Stealth[round]}, line width=0.4mm] (0.3,0) -- (-0.3,0);
		\end{tikzpicture}};
		
}

\newcommand\ArrowNL[3]
{
	\node at (#1+1, #2+0.5, #3-0.5) 
	%	\node at (#1, #2, #3) 
	{
		\begin{tikzpicture}
		\draw[->, thick, >={Stealth[round]}, line width=0.4mm] (0.16,0.16) -- (-0.16,-0.16);
		\end{tikzpicture}};
	
}

\newcommand\ArrowNR[3]
{
	\node at (#1+0.5, #2+1, #3-0.5) 
	%	\node at (#1, #2, #3) 
	{
		\begin{tikzpicture}
		\draw[->, thick, >={Stealth[round]}, line width=0.4mm] (0.16,0.16) -- (-0.16,-0.16);
		\end{tikzpicture}};
	
}

\newcommand\ArrowNW[3]
{
	\node at (#1+0.5, #2+0.5, #3) 
%	\node at (#1, #2, #3) 
	{
		\begin{tikzpicture}
		\draw[->, thick, >={Stealth[round]}, line width=0.4mm] (0,0.3) -- (0,-0.3);
		\end{tikzpicture}};
	
}

\newcommand\ArrowSW[3]
{
	\node at (#1+0.5, #2+0.5, #3) 
%	\node at (#1, #2, #3) 
	{
		\begin{tikzpicture}
		\draw[<-, thick, >={Stealth[round]}, line width=0.4mm] (0.3,0) -- (-0.3,0);
		\end{tikzpicture}};
	
}

\newcommand\ArrowSE[3]
{
	\node at (#1+0.5, #2+0.5, #3) 
%	\node at (#1, #2, #3) 
	{
		\begin{tikzpicture}
		\draw[<-, thick, >={Stealth[round]}, line width=0.4mm] (0,0.3) -- (0,-0.3);
		\end{tikzpicture}};
	
}

% Definition of \planepartition
% To draw the following plane partition, just write \planepartition{ {a, b, c}, {d,e} }.
%  a b c
%  d e
\newcommand\planepartition[1]{
	\setcounter{x}{-1}
	\foreach \a in {#1} {
		\addtocounter{x}{1}
		\setcounter{y}{-1}
		\foreach \b in \a {
			\addtocounter{y}{1}
			\setcounter{z}{-1}
			\foreach \c in {0,...,\b} {
				\addtocounter{z}{1}
				\ifthenelse{\c=0}{\setcounter{z}{-1},\addtocounter{y}{0}}{
					\cube{\value{x}}{\value{y}}{\value{z}}}
			}
		}
	}
}


%%\begin{tikzpicture}[x=(220:1cm), y=(-40:1cm), z=(90:0.707cm)]
	%\planepartition{{0,0,1},{1,1,1},{1,0,0},{1,0,1}};
\foreach \m [count=\y] in {{1,1,1,1},{1,1,1},{1,1,1,1},{1,1,1}}{
	\foreach \n [count=\x] in \m {
		\ifnum \n>0
		\foreach \z in {1,...,\n}{
			\draw [fill=blue!30] (\x+1,\y,\z) -- (\x+1,\y+1,\z) -- (\x+1, \y+1, \z-1) -- (\x+1, \y, \z-1) -- cycle;
			\draw [fill=blue!40] (\x,\y+1,\z) -- (\x+1,\y+1,\z) -- (\x+1, \y+1, \z-1) -- (\x, \y+1, \z-1) -- cycle;
			\draw [fill=blue!10] (\x,\y,\z)   -- (\x+1,\y,\z)   -- (\x+1, \y+1, \z)   -- (\x, \y+1, \z) -- cycle;  
		}
	
		\fi
	}
}   

\foreach \m [count=\y] in {{0,0,0,1}, {0,0,0}} {
	\foreach \n [count=\x] in \m {
		\ifnum \n>0
		\foreach \z in {1,...,\n}{
			\draw [fill=green!30] (\x+1,\y,\z) -- (\x+1,\y+1,\z) -- (\x+1, \y+1, \z-1) -- (\x+1, \y, \z-1) -- cycle;
			\draw [fill=green!40] (\x,\y+1,\z) -- (\x+1,\y+1,\z) -- (\x+1, \y+1, \z-1) -- (\x, \y+1, \z-1) -- cycle;
			\draw [fill=green!10] (\x,\y,\z)   -- (\x+1,\y,\z)   -- (\x+1, \y+1, \z)   -- (\x, \y+1, \z) -- cycle;  
		}
		
		\fi
	}
}   

\ArrowNW{1}	{3}	{1};

\foreach \m [count=\y] in {{0,0,0,0},{0,0,0},{0,0,0,0},{2,0,0}}{
	\foreach \n [count=\x] in \m {
		\ifnum \n>0
		\foreach \z in {1,...,\n}{
			\draw [fill=orange!30] (\x+1,\y,\z) -- (\x+1,\y+1,\z) -- (\x+1, \y+1, \z-1) -- (\x+1, \y, \z-1) -- cycle;
			\draw [fill=orange!40] (\x,\y+1,\z) -- (\x+1,\y+1,\z) -- (\x+1, \y+1, \z-1) -- (\x, \y+1, \z-1) -- cycle;
			\draw [fill=orange!10] (\x,\y,\z)   -- (\x+1,\y,\z)   -- (\x+1, \y+1, \z)   -- (\x, \y+1, \z) -- cycle;  
		}
		
		\fi
	}
}  

\foreach \m [count=\y] in {{0},{0},{0},{0,1,1}}{
	\foreach \n [count=\x] in \m {
		\ifnum \n>0
		\foreach \z in {1,...,\n}{
			\draw [fill=blue!30] (\x+1,\y,\z) -- (\x+1,\y+1,\z) -- (\x+1, \y+1, \z-1) -- (\x+1, \y, \z-1) -- cycle;
			\draw [fill=blue!40] (\x,\y+1,\z) -- (\x+1,\y+1,\z) -- (\x+1, \y+1, \z-1) -- (\x, \y+1, \z-1) -- cycle;
			\draw [fill=blue!10] (\x,\y,\z)   -- (\x+1,\y,\z)   -- (\x+1, \y+1, \z)   -- (\x, \y+1, \z) -- cycle;  
		}
		
		\fi
	}
} 

\foreach \m [count=\y] in {{0},{0},{0},{1,0,0}}{
	\foreach \n [count=\x] in \m {
		\ifnum \n>0
		\foreach \z in {1,...,\n}{
		%	\draw [fill=blue!30] (\x+1,\y,\z) -- (\x+1,\y+1,\z) -- (\x+1, \y+1, \z-1) -- (\x+1, \y, \z-1) -- cycle;
			\draw [fill=blue!40] (\x,\y+1,\z) -- (\x+1,\y+1,\z) -- (\x+1, \y+1, \z-1) -- (\x, \y+1, \z-1) -- cycle;
		%	\draw [fill=blue!10] (\x,\y,\z)   -- (\x+1,\y,\z)   -- (\x+1, \y+1, \z)   -- (\x, \y+1, \z) -- cycle;  
		}
		
		\fi
	}
} 
		"x" "y" "z"

\ArrowNR{4} {1}	{1};
\ArrowNL{4} {1}	{1};

\ArrowNR{4} {3}	{1};
\ArrowNL{4} {3}	{1};

\ArrowNR{3} {4}	{1};
\ArrowNL{3} {4}	{1};

\ArrowNR{1} {4}	{1};
\ArrowNR{2} {4}	{1};

\ArrowNL{3} {2}	{1};

\ArrowSW{1}	{1}	{1};
\ArrowSW{2}	{1}	{1};
\ArrowSW{3}	{1}	{1};
\ArrowSW{4}	{1}	{1};


\ArrowNW{1}	{2}	{1};
\ArrowSW{2}	{2}	{1};
\ArrowNW{3}	{2}	{1};

\ArrowNW{2}	{3}	{1};
\ArrowNE{3}	{3}	{1};
\ArrowNE{4}	{3}	{1};

%\ArrowNW{1}	{4}	{1};
\ArrowSW{2}	{4}	{1};
\ArrowNW{3}	{4}	{1};

%
%	\caption{large structure}
%
%	\label{fig:cover2}
%\end{figure}
%
%\begin{figure}[htb]
%
%  \centering
%  \includegraphics[width=3.4in]{Figures/cover.png}
%
%  \caption{M-Bocks modular robots with connections illuminated with onboard LEDs}
%
%  \label{fig:cover}
%\end{figure}
