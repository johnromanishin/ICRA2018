\section{Related Work}
\label{sec:RelatedWork}

Modular self-reconfigurable robots can been classified based on physical architecture: lattice, chain, or a hybrid of the two~\cite{Yim-RAM07}\cite{Moubarak2012}\cite{surveyyim}.  Many systems currently under development, including U-Bots~\cite{ubot}, Roombots~\cite{roombots3}, and SMORES~\cite{Yim-RAM07}, utilize a hybrid architecture. The fundamental distinction between hybrid or chain modules and strict lattice systems is that hybrid or chain modules have either fewer connector faces than lattice faces, or these connector faces are located in off-lattice positions.  Chain and hybrid systems are typically designed to self-reconfigure using complicated implementations which approximate simpler models, such as the sliding cube model~\cite{FitchRus-IROS03} or the pivoting cube model~\cite{RomanishinRus-IROS13}.

\subsection{Modular Robots Overview}

\subsection{Location Tagging technology overview}

\begin{table*}[ht]
	\centering
	\caption{Comparison of attributes for several various tagging technologies}
	\newcommand{\wdd}{1.8cm}
	\begin{tabular}{ p{\wdd} |p{\wdd}  p{\wdd} p{\wdd} p{\wdd} p{\wdd} p{\wdd} p{\wdd}  }
		\hline
							%1 - EMPTY
		& RFID 				%2
		& Light Based		%3
		& NFC Tags 			%4
		& Electrical 		%5s
		& QR Codes 			%6
		& Barcodes			%7
		& \tagNamePlural \\ %8
		\hline
%%		   				RFID		LIGHT		NFC		ELEC	QR		
		%%		1			2		3			4		5		6		7		8
		Description		& Expensive	& 			& 		& 		& 		& 	  	& Inexpensive \\
		Cost			& Expensive	& 			& 		& 		& 		& 	  	& Inexpensive \\
		Size 			&  			& 			&    	&   	&       &     	& Small		  \\
		Passive 		&  			&			&  		&	 	&		& 		& yes!		  \\
		Orientation 	&  			& 			&  		&	 	&	  	& 		& yes!		\\
		Modular Robots 	&  			& yes		&	  	&		&		& 		& 3D M-Blocks\\
		Range			& 0-10$m$	&			&		&		&		&		& 0-1$mm$	\\
	\end{tabular}
	\label{tab:tagTech}    
\end{table*}

There are several different types of passive tagging technologies designed allow the recognization of tags. These include RFID, NFC, QR-Codes, April Tags, etc... 


