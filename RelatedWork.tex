
%%%%%%%%%%%%%%%%%%%%%%%%%%%%%%%%%%%%%%%%%%%%%%%%%%%%%%%%%%%%%%%%%%%%%%%%%%%%%%%%%%%%%%%%%%%%%%%%%%%%%%%%%%%%%%%%%%%%%%%%%%%%%%%%%%%%%%%%%%%%%%%
%%%%%%%%%%%%%%%%%%%%%%%%%%%%%%%%%%%%%%%%%%%%%%%%%%%%%%%%%%%%%%%%%%%%%%%%%%%%%%%%%%%%%%%%%%%%%%%%%%%%%%%%%%%%%%%%%%%%%%%%%%%%%%%%%%%%%%%%%%%%%%%
\section{Related Work}
\label{sec:RelatedWork}

\begin{table*}[h]
	\centering
	\caption{Comparison of attributes for several tagging technologies utilized by MSRR systems in order to determine the configuration of assemblies of modules.}
	\newcommand{\wdd}{2.0cm}
	\begin{tabular}{ p{2.4 cm} p{\wdd}  p{\wdd} p{\wdd} p{\wdd} p{\wdd} p{\wdd} p{\wdd}  }
		\hline
		\addlinespace[1ex]
		%1 - EMPTY
																				& RFID / NFC		& Optical				& Electrical 					& QR Codes 			& Inductive				& \bf{\tagNamePlural} 		\\ %8
		\hline
		
		\addlinespace[0.5ex]	\textit{Information Storage or transfer medium}	& radio waves		& IR or visible light	& direct wired connections		& 2D optical grid	& inductive				& permanent Magnet Field	\\
		
		\addlinespace[0.5ex]	\textit{Tag Cost}								& inexpensive		& moderate				& inexpensive					& cheap	 			& moderate				& inexpensive 				\\
		
		\addlinespace[0.5ex]	\textit{Reader Cost}							& expensive			& moderate				& moderate						& expensive 	 	& moderate				& moderate 					\\
		
	%	\addlinespace[0.5ex] 	\textit{Size} 			& significant 		& minimal 				& minimal	  		& small      		& varies				& Small		  				\\
		
		\addlinespace[0.5ex]	\textit{Passive} 								& yes				& no					& possible	 					& yes				& no					& yes		  				\\
		
		\addlinespace[0.5ex]	\textit{Communication} 							& possible			& yes					& yes	 						& no				& yes					& not yet		  			\\
		
		\addlinespace[0.5ex] 	\textit{Orientation} 							& needs 4 tags 		& possible 				& possible	 					& yes				& needs 4+ points		& Yes						\\
		
		\addlinespace[1ex] 	\textit{MSRR Systems}
		& \cite{StigmergyWerfel2006}	%RFID	
		& CK-Bot~\cite{park2008automatic}					%OPTICAL
		& \cite{Soldercubes2016}, ~\cite{ubot-Zhu-2014}	%Electrical
		& \cite{lin2017vision}								%QR Codes 
		& SMORES~\cite{TosunDaveyLiuYim-IROS2016}, Catoms~\cite{Kirby-IROS07}					%Inductive
		& 3D M-Blocks~\cite{Romanishin20153d}	\\ 			%M-Tags
		
		\addlinespace[1ex] 	\textit{Range}										& 0-10+$m$		& variable				& 0 $m$								& 1$mm$ - 1$m$ +		& 0-10$mm$				& 0-1$mm$	\\
	\end{tabular}
	\label{tab:tagTech}
\end{table*}

%%%%%%%%%%%%%%%%%%%%%%%%%%%%%%%%%%%%%%%%%%%%%%%%%%%%%%%%%%%%%%%%%%%%%%%%%%%%%%%%%%%%%%%%%%%%%%%%%%%%%%%%%%%%%%%%%%%%%%%%%%%%%%%%%%%%%%%%%%%%%%%
%%%%%%%%%%%%%%%%%%%%%%%%%%%%%%%%%%%%%%%%%%%%%%%%%%%%%%%%%%%%%%%%%%%%%%%%%%%%%%%%%%%%%%%%%%%%%%%%%%%%%%%%%%%%%%%%%%%%%%%%%%%%%%%%%%%%%%%%%%%%%%%

Many papers provide a comprehensive overview of the MSRR field including the seminal article published in 2007~\cite{Yim-RAM07} in addition to several more recent updates, including~\cite{chennareddy2017modular} which focuses on the hardware systems, and ~\cite{abukhalil2013survey} which focuses on algorithmic developments. In this section of the paper we investigate two different topics in the related work, (A) the challenge of configuration discovery in modular robotics systems, and (B) prior research on behaviors for MSRR systems.

%%%%%%%%%%%%%%%%%%%%%%%%%%%%%%%%%%%%%%%%%%%%%%%%%%%%%%%%%%%%%%%%%%%%%%%%%%%%%%%%%%%%%%%%%%%%%%%%%%%%%%%%%%%%%%%%%%%%%%%%%%%%%%%%%%%%%%%%%%%%%%%
%%%%%%%%%%%%%%%%%%%%%%%%%%%%%%%%%%%%%%%%%%%%%%%%%%%%%%%%%%%%%%%%%%%%%%%%%%%%%%%%%%%%%%%%%%%%%%%%%%%%%%%%%%%%%%%%%%%%%%%%%%%%%%%%%%%%%%%%%%%%%%%
\subsection{Configuration Discovery in MSRR Systems}
\label{ssec:RWconfiguration}
%%%%%%%%%%%%%%%%%%%%%%%%%%%%%%%%%%%%%%%%%%%%%%%%%%%%%%%%%%%%%%%%%%%%%%%%%%%%%%%%%%%%%%%%%%%%%%%%%%%%%%%%%%%%%%%%%%%%%%%%%%%%%%%%%%%%%%%%%%%%%%%
%%%%%%%%%%%%%%%%%%%%%%%%%%%%%%%%%%%%%%%%%%%%%%%%%%%%%%%%%%%%%%%%%%%%%%%%%%%%%%%%%%%%%%%%%%%%%%%%%%%%%%%%%%%%%%%%%%%%%%%%%%%%%%%%%%%%%%%%%%%%%%%
The ability for a modular system to discover the specific configuration of modules at any and all times is one of the most essential tasks for any modular robotic system. However, many of the modular robotic systems proposed to date have either built very limited and error prone methods for solving this problem, or have omitted implementation of such a system entirely. Any \textit{connection} between modules in a modular system involves a minimum of three unique parameters (1) some form of identity or \textit{ID} of the two modules in question, (2) the connector, or \textit{face number} for each of the modules, and 3) the \textit{orientation} of the connection, which in a cubic lattice is one of four possible 90 degree values. Much of the algorithmic work which concerns configuration discover including~\cite{park2008automatic} and in~\cite{Funiak-IJRR09}, assumes that this information is accurately determined. However these works focus more on the algorithmic challenges, abstracting away some of the details of implementing the algorithms.

Connectors are one of the most significant design challenges in creating practical MSRR systems. Aside from their fundamental requirement of providing robust mechanical links, these connectors have been used in the literature to enable inter-module communication \cite{liedke2013collective}, \cite{TosunDaveyLiuYim-IROS2016}, deliver and share power \cite{OptimalPowerSharing2016}, and determine the presence and relative orientation of adjacent modules. However virtually all of the solutions for gathering presence and orientation information in MSRR require active communication between two robots, e.g. the inductive links in the SMORES systems~\cite{TosunDaveyLiuYim-IROS2016} and CATOMS~\cite{Kirby-IROS07}, or IR optical links used in many of the early MSRR, including MTRAN~\cite{Kurokawa-IJRR08}. An ideal neighbor detector would work passively, essentially acting like a barcode. This would increase the robustness and scalability of the modular system, since it would allow passive elements to interact with active elements. There is a rich potential array of technologies which might be used to implement neighbor detection connectors, shown in Table~\ref{tab:tagTech}.

While there are many potential solutions that provide passive detection, RFID or NFC tags are quite prevalent in robotics research. RFID tags are implemented in several robots in fields related to MSRR, including~\cite{werfel2006distributed}, but due to the size and cost of the reader electronics these are impractical for MSRR systems with smaller characteristic dimensions. NFC is a technology related to RFID that looks promising for this application, but it remains a proprietary standard, and is not easily accessible at the present time. Additional passive methods, such as QR tags or April Tags~\cite{wang2016iros} are effective in other fields of robotics, but require the use of a camera and sophisticated image processing techniques which at the moment are impractical to include in every active connector of a MSRR system.

%%%%%%%%%%%%%%%%%%%%%%%%%%%%%%%%%%%%%%%%%%%%%%%%%%%%%%%%%%%%%%%%%%%%%%%%%%%%%%%%%%%%%%%%%%%%%%%%%%%%%%%%%%%%%%%%%%%%%%%%%%%%%%%%%%%%%%%%%%%%%%%
%%%%%%%%%%%%%%%%%%%%%%%%%%%%%%%%%%%%%%%%%%%%%%%%%%%%%%%%%%%%%%%%%%%%%%%%%%%%%%%%%%%%%%%%%%%%%%%%%%%%%%%%%%%%%%%%%%%%%%%%%%%%%%%%%%%%%%%%%%%%%%%
\subsection{MSRR algorithms}
\label{ssec:RW-Algorithmic}
%%%%%%%%%%%%%%%%%%%%%%%%%%%%%%%%%%%%%%%%%%%%%%%%%%%%%%%%%%%%%%%%%%%%%%%%%%%%%%%%%%%%%%%%%%%%%%%%%%%%%%%%%%%%%%%%%%%%%%%%%%%%%%%%%%%%%%%%%%%%%%%
%%%%%%%%%%%%%%%%%%%%%%%%%%%%%%%%%%%%%%%%%%%%%%%%%%%%%%%%%%%%%%%%%%%%%%%%%%%%%%%%%%%%%%%%%%%%%%%%%%%%%%%%%%%%%%%%%%%%%%%%%%%%%%%%%%%%%%%%%%%%%%%
While there have been many simulated algorithms and control hierarchies that have been presented which accomplish distributed behaviors, e.g. ~\cite{Jones-ICRA03, butler2002generic, stoy-747simulation-2004} most of these works abstract away various challenges that real-world modules would face. There have been several works which present decenteralized algorithms operating on actual hardware, including the UBot~\cite{zhu2015simplified}, the ATRON system~\cite{christensen2013distributed}, and several others, e.g. \cite{germanswarm-Levi-2014, Yim-IROS07}. However few of these systems provide a clear path to being able to reconfigure according to a generalized 3D movement framework, which limits the reconfigurability and scalability of these systems. There are many works introducing various algorithms and control strategies similar to those we propose in this paper, and we are not claiming the behaviors we present are novel. This work from 2014~\cite{abukhalil2013survey}, provides an overview of some of the existing academic work involving decentralized control strategies for MSRR and robot swarms.

Light tracking behaviors can be traced to the canonical Braitenberg vehicles~\cite{braitenberg1986vehicles}, and are implemented in simulation on M-Block - like hardware in~\cite{sclaicithesis2016} and on previous 3D M-Blocks~\cite{claici2017distributed}. Other MSRR which we are aware of include the Polybot~\cite{Yim-IROS07}, and many non-modular systems, e.g. the Kilobots~\cite{Rubenstein-ICRA12} have attempted to aggregate using light signals include addition to several papers detailing control strategies and algorithms for line formation in MSRR systems~\cite{claici2017distributed, sclaicithesis2016}.


%	For modular robots, knowing which neighbors each module is connected to and the relative angle of that connection is indispensable for MSRR systems. While systems have left this problem un-addressed, most MSRR systems have some method to determine the configuration. Some connections also implement inter-module communication between modules using the same hardware used for configuration detection. Other systems use passive methods like RFID tags, e.g.\cite{Werfel-PhDThesis06} to sense module adjacency. 
%inclusion of ``passive'' modules in the robot as described by \cite{roombots-Bonardi-2013}. 
% Other systems use RFID tags \cite{Werfel-PhDThesis06} to explicitly sense module adjacency.
% implicitly sense adjacent neighbors \cite{liedke2013collective} \cite{Gilpin-Thesis06} \cite{TosunDaveyLiuYim-IROS2016}. 
%%%%%%%%%%%%%%%%%%%%%%%%%%%%%%%%%%%%%%%%%%%%%%%%%%%%%%%%%%%%%%%%%%%%%%%%%%%%%%%%%%%%%%%%%%%%%%%%%%%%%%%%%%%%%%%%%%%%%%%%%%%%%%%%%%%%%%%%%%%%%%%%
%%%%%%%%%%%%%%%%%%%%%%%%%%%%%%%%%%%%%%%%%%%%%%%%%%%%%%%%%%%%%%%%%%%%%%%%%%%%%%%%%%%%%%%%%%%%%%%%%%%%%%%%%%%%%%%%%%%%%%%%%%%%%%%%%%%%%%%%%%%%%%%%
%\subsection{Location Tagging technology overview}
%\label{ssec:RWtaggingTech}
%%%%%%%%%%%%%%%%%%%%%%%%%%%%%%%%%%%%%%%%%%%%%%%%%%%%%%%%%%%%%%%%%%%%%%%%%%%%%%%%%%%%%%%%%%%%%%%%%%%%%%%%%%%%%%%%%%%%%%%%%%%%%%%%%%%%%%%%%%%%%%%%
%%%%%%%%%%%%%%%%%%%%%%%%%%%%%%%%%%%%%%%%%%%%%%%%%%%%%%%%%%%%%%%%%%%%%%%%%%%%%%%%%%%%%%%%%%%%%%%%%%%%%%%%%%%%%%%%%%%%%%%%%%%%%%%%%%%%%%%%%%%%%%%%
%
%There are several different types of passive tagging technologies designed allow a reader to extract unique identification numbers from tags. These include RFID, NFC, QR-Codes, April Tags~\cite{wang2016iros}. This section will provide a brief discussion of the characteristics of each of these technologies, and their suitability to working in the context of cm module scale MSRR systems. Table~\ref{tab:tagTech} provides an over and a (subjective) comparison between several of these technologies, and a listing of several example multi-robot systems which utilize this approach for broadly similar configuration detection roles.