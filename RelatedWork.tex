\section{Related Work}
\label{sec:RelatedWork}

Modular self-reconfigurable robots are often characterized by their
system topology: lattice, chain, or hybrid~\cite{Yim-RAM07}.  Most of
the systems currently under development including U-Bots~\cite{ubot},
Roombots~\cite{roombots3}, and SMORES~\cite{Yim-RAM07} utilize a hybrid
architecture. The fundamental distinction between hybrid or chain modules and
strict lattice systems is that hybrid or chain modules have either fewer
connector faces than lattice faces, or these connector faces are
located in off-lattice positions.  Chain and hybrid
systems are typically designed to self-reconfigure using complicated
implementations which approximate simpler models, such as the sliding cube model~\cite{FitchRus-IROS03} or
the pivoting cube model~\cite{RomanishinRus-IROS13}.

\subsection{Tagging technology overview}

There are several different types of passive tagging technologies designed allow the recognization of tags. These include RFID, NFC, QR-Codes, April Tags, etc... 

%There have been several systems which attempt to implement the sliding
%cube model~\cite{Hosokawa-ICRA98, Suzuki-IROS07, An-ICRA08}, but these systems
%have been limited to two dimensions.  Additionally, there are systems
%which are able to self-reconfigure in three
%dimensions~\cite{Kurokawa-IJRR08, Kurokawa-IROS98}, but these systems
%all diverge from the simplicity offered by the sliding and pivoting
%cube models.  We are not aware of any preexisting three dimensional
%hardware that is able to reconfigure in a manner that directly mimics
%the theoretical models.  Furthermore, recent
%research~\cite{Sacristan-EuroCG2013} has produced a provably correct
%self-reconfiguration algorithm for two dimensional systems based on
%the sliding cube model. No such solution in three dimensions has yet been presented for the sliding
%cube model or for the pivoting cube model. 
%
%Many existing modular systems are also dependent on complex,
%mechanically active connectors which require careful
%alignment~\cite{ubot, roombots3, Yim-RAM07, WolfeChirikjian-ICRA12}.  In
%contrast, the M-Blocks use passive magnetic connectors which
%automatically self-align.  While these magnetic connectors may not be
%as strong as protruding mechanical latches found in other systems, they
%are simple to use.  Continuing advancements in advanced connector
%design, such as solder-based connectors~\cite{NeubertLipson-TR2014} may provide
%additional options for the M-Blocks in the future.
%
%Most existing modular systems are also limited by the inability of the
%modules to move independently.  Like the M-Blocks, a few other system
%do not suffer from this drawback, for example
%$M^{3}$Express~\cite{WolfeChirikjian-ICRA12} whose wheeled modules can
%drive without being a part of a larger group of modules.  The
%Cubli~\cite{Cubli} is a recently developed robot which uses
%torque-producing flywheels to move and balance on its edges.  It can
%also move independently, but unlike the M-Blocks, Cubli is not
%designed to operate in large ensembles and cannot climb over and
%around other modules.

