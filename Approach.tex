%%%%%%%%%%%%%%%%%%%%%%%%%
\section{Approach}
\label{sec:Approach}
%%%%%%%%%%%%%%%%%%%%%%%%%

Modular self-reconfigurable robots have been classified based on physical architecture: lattice, chain, or a hybrid of the two.~\cite{Yim-RAM07}\cite{Moubarak2012}\cite{surveyyim} Lattice architectures maintain their shape by fully constraining individual modules to one another, and in general a specific topology of inter-modular connections implies a specific shape of the system. Relative orientation of connecting faces is essential to manipulation and modeling of lattice architectures. An inexpensive, precise, orientation-sensitive ID tag would greatly assist in the development of systems with many connecting faces, where configuration is manipulated by explicit changes in topology. A precise tag with sensitivity to orientation would be equally beneficial in chain architectures, for a different reason. Articulating joints in a chain architecture allow module positions to fall outside of lattice points, meaning that maintaining a model of the total configuration, or finely controlling a move, both require close awareness of orientation between modules. 