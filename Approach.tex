%%%%%%%%%%%%%%%%%%%%%%%%%
\section{Approach}
\label{sec:Approach}
%%%%%%%%%%%%%%%%%%%%%%%%%

Modular self-reconfigurable robots have been classified based on physical architecture: lattice, chain, or a hybrid of the two.~\cite{Yim-RAM07}\cite{Moubarak2012}\cite{surveyyim} Lattice architectures take shape by fully constraining individual modules to one another, and there tend to be several connecting faces in a given configuration due to these constraints. Interpreting relative orientation of the connecting faces is essential to modeling and manipulation of the overall configuration, because changes in shape are achieved via changes in topology, i.e. by explicitly moving modules and changing how they are connected. An inexpensive, precise, orientation-sensitive ID tag would greatly assist in the development of autonomous systems with vast numbers of connecting faces.

A precise tag with sensitivity to orientation would be equally essential in chain architectures. Articulation of joints in a chain architecture allows module position to be oriented outside the constraints of lattice points. Moves in a non-grid-oriented space like this one, especially moves that articulate at multiple joints or propagate through multiple modules, require that the orientation between module faces be clearly defined in order to sustain a model of the total configuration.

The ideal connective position sensor would be appropriate in either use case. It would be inexpensive, to help satisfy the ease of scalability in a modular robotics system where there may be hundreds of individual modules. It would also be precise, because individual errors in a repeated sequence of connections would be multiplicative and thus damaging to the concerted operation of a whole configuration of modules. In addition, it would not be enough to merely identify an object or even to detect which connecting faces are adjacent in the lattice or chain, because effective concerted motion between modules may require them to identify how they and their neighbors are oriented with respect to the larger configuration.

The approach that \tagNamePlural take towards this problem is a passive magnetic connector that uses a combination of dipole angles to communicate identity and relative rotational position. In the Mblocks implementation, each face of each cube is encoded with a different combination of dipole angles than every other face of every other cube. Magnetic rotary position sensors read these dipole angle combinations to determine which face is directly adjacent, and they identify the rotational angle of the face as well. All of this occurs using an unpowered array of magnets that provide static EM fields, readable at a hyperlocal distance.

The underlying hypothesis is that low cost and lack of RF transmission make magnets an attractive choice for a technology that may have to be implemented hundreds of times in a set of reconfigurable modules. The ability to use several such sensors at close proximity allows several to be used on each connecting face, providing enough data to read configuration as well as identity.